

\documentclass[a4paper,12pt]{report}

\usepackage{graphicx}
\usepackage{amssymb}
\usepackage{amsmath}
\usepackage{amsfonts}

\usepackage{placeins}


\usepackage[english]{babel}
\usepackage{booktabs}
\usepackage{stfloats}
\usepackage[T1]{fontenc}

\usepackage[titletoc,title]{appendix}

\usepackage{float}
\restylefloat{table}

\usepackage{longtable}

\usepackage{xcolor,colortbl}

\usepackage{mathpazo} % Palatino
\usepackage{avant}    % Avant Garde

\usepackage[margin=2.6cm, bottom=3.cm, top=3.cm, portrait]{geometry}\usepackage{caption}\usepackage[a4paper]{hyperref}\usepackage{adjustbox}\usepackage{enumitem}

 \usepackage[stable]{footmisc}

\renewcommand{\vec}[1]{\mathbf{#1}}

\setlength\parindent{0pt}

\newcommand{\HRule}[1]{\hfill \rule{0.2\linewidth}{#1}}

\makeatletter
\newcommand\primitiveinput[1]
{\@@input #1 }
\makeatother

\definecolor{Gray}{gray}{0.90}

\renewcommand\thesection{\arabic{section}}

%\renewcommand\appendixname{Appendix}
%\renewcommand\appendixpagename{Appendix}

\begin{document}


\thispagestyle{empty} % Remove page numbering on this page

%----------------------------------------------------------------------------------------
%	TITLE SECTION
%----------------------------------------------------------------------------------------

\colorbox{Gray}{
	\parbox[t]{1.0\linewidth}{
		\centering \fontsize{28pt}{40pt}\selectfont % The first argument for fontsize is the font size of the text and the second is the line spacing - you may need to play with these for your particular title
		\vspace*{0.7cm} % Space between the start of the title and the top of the grey box
		
		\hfill PyPIC \\
		\hfill physics manual

%		\hfill Beam Screens\\
		
		\vspace*{0.7cm} % Space between the end of the title and the bottom of the grey box
	}
}

%----------------------------------------------------------------------------------------

%\vfill
%                \hfill {\huge Bunch length 1.0 ns}
                \vfill

%----------------------------------------------------------------------------------------
%	AUTHOR NAME AND INFORMATION SECTION
%----------------------------------------------------------------------------------------

{\centering \large 
\hfill Giovanni Iadarola \\
%\hfill \\
\hfill CERN - Geneva, Switzerland

\HRule{1pt}} % Horizontal line, thickness changed here

%----------------------------------------------------------------------------------------

\clearpage % Whitespace to the end of the page

\newpage

\renewcommand*{\arraystretch}{1.4}



\section{Space charge}



We assume that the bunch travels rigidly along $s$ with velocity $\beta_0 c$:
\begin{align}
&\rho\left(x, y, s, t\right) = \rho_0\left(x, y, s - \beta_0 ct\right) \label{rhorho0}\\
&\textbf{J}\left(x, y, s, t\right) = \beta_0c\, \rho_0\left(x, y, s - \beta_0 ct\right)  \hat{\textbf{i}}_s \label{JJ0}
\end{align}

We define an auxiliary variable $\zeta$ as the position along the bunch:
\begin{equation}
\zeta = s -\beta_0 c t \, .\label{zetadef}
\end{equation}
We call $K$ the lab reference frame in which we have defined all equations above, and we introduce a boosted frame $K'$ moving rigidly with the reference particle.
The coordinates in the two systems are related by a Lorentz transformation~\cite{jackson}:
\begin{align}
ct' &= \gamma_0 \left(ct -\beta_0 s \right)\label{lorA}\\
x' &= x\label{lorX}\\
y' &= y\label{lorY}\\
s' &= \gamma_0 \left(s -\beta_0 ct \right) = \gamma_0 \zeta\label{lorB}
\end{align}
The  corresponding inverse transformation is:
\begin{align}
ct &= \gamma_0 \left(ct' +\beta_0 s' \right)\label{lorC}\\
x &= x'\label{lorXinv}\\
y &= y'\label{lorYinv}\\
s &= \gamma_0 \left(s' +\beta_0 ct' \right)\label{lorD}
\end{align}



The quantities $\left(c \rho, J_x, J_y, J_s\right)$ form a Lorentz 4-vector and therefore they are transformed between $K$ and $K'$ by relationships similar to the Eqs.~\ref{lorA}-\ref{lorY}~\cite{jackson}:
\begin{align}
c\rho' \left(\textbf{r'}, t'\right)\ &= \gamma_0 \left[c \rho  \left(\textbf{r}\left(\textbf{r'}, t'\right), t\left(\textbf{r'}, t'\right)\right) -\beta_0 J_s \left(\textbf{r}\left(\textbf{r'}, t'\right), t\left(\textbf{r'}, t'\right)\right) \right]\label{lorrho}\\
J_s' \left(\textbf{r'}, t'\right)\ &= \gamma_0 \left[J_s  \left(\textbf{r}\left(\textbf{r'}, t'\right), t\left(\textbf{r'}, t'\right)\right) -\beta_0 c \rho \left(\textbf{r}\left(\textbf{r'}, t'\right), t\left(\textbf{r'}, t'\right)\right) \right]\label{lorjs}
\end{align}
where the transformations $\textbf{r}\left(\textbf{r'}, t'\right)$ and $t\left(\textbf{r'}, t'\right)$ are defined by Eqs.~\ref{lorC} and~\ref{lorD} respectively. The transverse components $J_x$ and $J_y$ of the current vector are invariant for our transformation, and are anyhow zero in our case.

Using Eq.\,\ref{JJ0} these become:
\begin{align}
\rho' \left(\textbf{r'}, t'\right)\ &= \frac{1}{\gamma_0}\rho\left(\textbf{r}\left(\textbf{r'}, t'\right), t\left(\textbf{r'}, t'\right)\right)
\\
J_s' \left(\textbf{r'}, t'\right)\ & = 0
\end{align}

Using Eqs.~\ref{rhorho0} and~\ref{lorC}-\ref{lorYinv}, we obtain:
\begin{equation}
\rho  \left(x', y', s(s', t'), t(s', t')\right) = \rho_0  \left(x', y', s(s', t') - \beta_0 c\,t(s', t')\right)
\end{equation}

From Eq.~\ref{lorB} we get:
\begin{equation}
s(s', t')- \beta_0 c\,t(s', t') = \frac{s'}{\gamma_0} 
\end{equation}
where the coordinate $t' $ has disappeared.

We can therefore write:
\begin{equation}
\rho' \left(x', y', s', t'\right) =   \frac{1}{\gamma_0} \rho_0  \left(x', y',  \frac{s'}{\gamma_0}\right)\label{rhoprimerho0}
\end{equation}

The electric potential in the bunch frame is solution of Poisson's equation:

\begin{equation}
\frac{\partial^2 \phi'}{\partial x'^2} +  \frac{\partial^2 \phi'}{\partial y'^2}+  \frac{\partial^2 \phi'}{\partial s'^2}= -\frac{\rho' (x', y', s')}{\varepsilon_0}
\end{equation}

From Eq.~\ref{rhoprimerho0} we can write:
\begin{equation}
\frac{\partial^2 \phi'}{\partial x'^2} +  \frac{\partial^2 \phi'}{\partial y'^2}+  \frac{\partial^2 \phi'}{\partial s'^2}= -\frac{1}{\gamma_0\varepsilon_0}  \rho_0 \left(x', y', \frac{s'}{\gamma_0}\right)\label{poissrho0}
\end{equation}

We now make the substitution:
\begin{equation}
\zeta = \frac{s'}{\gamma_0} \label{subst}
\end{equation}
obtained from Eq.~\ref{lorB}, which allows to rewrite Eq.~\ref{poissrho0} as:
\begin{equation}
\frac{\partial^2 \phi'}{\partial x^2} +  \frac{\partial^2 \phi'}{\partial y^2}+  \frac{1}{\gamma_0^2}\frac{\partial^2 \phi'}{\partial \zeta^2}=  -\frac{1}{\gamma_0\varepsilon_0}{\rho}_0 \left(x, y,\zeta\right) \label{modifpoiss}
\end{equation}
Here we have dropped the ``$'$'' sign from $x$ and $y$ as these coordinates are unaffected by the Lorentz boost.


\textbf{For large enough values of $\gamma$, Eq.~\ref{modifpoiss} can be approximated by:}
\begin{equation}
\frac{\partial^2 \phi'}{\partial x^2} +  \frac{\partial^2 \phi'}{\partial y^2} = -\frac{1}{\gamma_0\varepsilon_0}{\rho}_0 \left(x, y,\zeta\right) \label{2dpoiss}
\end{equation}

The quantities $\left( \frac{\phi}{c}, A_x, A_y, A_s\right)$ form a Lorentz 4-vector, we can show that the $s$ component of the vector potential in the lab frame vanishes:
\begin{align}
\phi &= \gamma_0 \left( {\phi'} +  \beta_0 c A'_s\right)\\
A_s &= A'_s +\beta_0 \frac{\phi'}{c}
\end{align}
In the bunch frame the charges are at rest therefore $A'_x=A'_y=A'_z=0$ therefore:
\begin{align}
\phi &= \gamma_0 \phi'\label{phiphip}\\
A_s &= \beta_0 \frac{\phi'}{c} =  \frac{\beta_0}{\gamma_0c}\phi
\end{align}

Combining Eq.\,\ref{phiphip} with Eq.\,\ref{modifpoiss} we obtain the equation in $\phi$:
\begin{equation}
\frac{\partial^2 \phi}{\partial x^2} +  \frac{\partial^2 \phi}{\partial y^2}+  \frac{1}{\gamma_0^2}\frac{\partial^2 \phi}{\partial \zeta^2}=  -\frac{1}{\varepsilon_0}{\rho}_0 \left(x, y,\zeta\right) \label{modifpoiss}
\end{equation}

%\section{Interaction time}
%In the lab frame the particle moves with speed $\beta$:
%\begin{equation}
%s(t) = \zeta_p +\beta c t
%\end{equation}
%
%In the frame $K'$, the kinematic equation of the particle can be obtained by replacing Eqs.~\ref{lorC} and~\ref{lorD} into Eq.~\ref{st_tau}:
%\begin{equation}
%\gamma_0 \left(s' +\beta_0 ct' \right) = \zeta_p +\beta \gamma_0 \left(ct' +\beta_0 s' \right)
%\end{equation}
%
%Solving for $s'$ we obtain:
%\begin{equation}
%s' = -\beta \gamma c \tau = \gamma \zeta\label{sprimezeta}
%\end{equation}
%Of course for the reference particle we obtain $s' = 0$.
%We observe that \textbf{beam particles are at rest in the reference frame $K'$ and that the distance between them is increased by a factor $\gamma$ with respect to the lab frame $K$}.

%\section{Transverse kick on the beam particle}
%
%We now evaluate the change on the transverse momentum for a beam particle defined in the lab frame by its transverse coordinates $x$ and $y$ and by its delay $\tau$ with respect to the reference particle (or equivalently by its $\zeta$ coordinate, defined by Eq.~\ref{zetadef}).
%
%We have seen that in the frame $K'$ the particle is at rest and has longitudinal coordinate $s' = \gamma \zeta$ (see Eq.~\ref{sprimezeta}). 
%The x' component of the electric field $\textbf{E}'$ acting on P is given by (see Eqs.~\ref{potential} and~\ref{phiphiprime}):
%\begin{equation}
%E'_x = -\frac{\partial \phi'}{\partial x} = -\frac{1}{\gamma_0}\frac{\partial \phi}{\partial x} \label{Exprime}
%\end{equation}
%Again, we have dropped the ``$'$'' sign from $x$ and $y$ as these coordinates are unaffected by the Lorentz boost.
%
%
%The change in the x component of the momentum, which is an invariant for our Lorentz transformation, is given by :
%\begin{equation}
%\Delta P_x = \Delta P'_x = qE'_x T'
%\end{equation}
%
%Using Eqs.~\ref{Exprime} and~\ref{Tprime} we can write:
%\begin{equation}
%\Delta P_x = -\frac{qL}{\beta c} \frac{\partial \phi}{\partial x}\left(x, y,\zeta\right)
%\end{equation}
%
%Normalizing to the momentum of the reference particle:
%
%\begin{equation}
%\Delta p_x = \frac{\Delta P_x} {P}= -\frac{qL}{ m\gamma\beta^2 c^2} \frac{\partial \phi}{\partial x}\left(x, y,\zeta\right)\label{dpx}
%\end{equation}
%
%Similarly, for the $y$-direction we can write: 
%\begin{equation}
%\Delta p_y = \frac{\Delta P_y} {P}= -\frac{qL}{ m\gamma\beta^2 c^2} \frac{\partial \phi}{\partial y}\left(x, y,\zeta\right)\label{dpy}
%\end{equation}


\section{Lorentz force}


We stay in the thin lens approximation so we approximate the velocity vector of the particle as:
\begin{equation}
\textbf{v} = \beta c\, \hat{\textbf{i}}_s
\end{equation}

We want to compute the Lorentz force acting on the particle:
\begin{equation}
\begin{split}
\textbf{F} &=q \left( -\nabla \phi -\frac{\partial \textbf{A}}{\partial t}
 + \beta c \ \hat{\textbf{i}}_s \times {\left(\nabla \times \textbf{A} \right)} \right)\\
 &=q \left( -\nabla \phi -\frac{\beta_0}{\gamma_0 c}\frac{\partial \phi}{\partial t}\hat{\textbf{i}}_s
 + \beta c \ \hat{\textbf{i}}_s \times {\left(\nabla \times \textbf{A} \right)} \right)
 \end{split}
\end{equation}

We compute the vector product:
\begin{align}
\begin{split}
\hat{\textbf{i}}_s \times \left(\nabla \times \textbf{A}\right) &= \left(\frac{\partial A_s}{\partial x} - \frac{\partial A_x}{\partial s} \right) \hat{\textbf{i}}_x + \left(\frac{\partial A_s}{\partial y} - \frac{\partial A_y}{\partial s} \right) \hat{\textbf{i}}_y\\
 &= \left(\frac{\partial A_s}{\partial x} - \frac{\partial A_x}{\partial s} \right) \hat{\textbf{i}}_x + \left(\frac{\partial A_s}{\partial y} - \frac{\partial A_y}{\partial s} \right) \hat{\textbf{i}}_y + \underbrace{\left(\frac{\partial A_s}{\partial s} - \frac{\partial A_s}{\partial s} \right)}_{=0} \hat{\textbf{i}}_s\\
 &= \nabla A_s - \frac{\partial \textbf{A}}{\partial s} 
\end{split} 
\end{align}

We replace:
\begin{equation}
\textbf{F} 
=q \left( -\nabla \phi -\frac{\beta_0}{\gamma_0 c}\frac{\partial \phi}{\partial t}\hat{\textbf{i}}_s
 + \beta  \beta_0\nabla \phi - \frac{\beta \beta_0}{\gamma_0} \frac{\partial \phi}{\partial s} \hat{\textbf{i}}_s
  \right)
\end{equation}

The potentials will have the same form as the sources (this can be shown explicitly using the Lorentz transformations):
\begin{equation}
\phi(x, y, s, t) = \phi_0\left(x, y, t - \frac{s}{\beta_0 c}\right)
\end{equation}
For a function in this form we can write:
\begin{equation}
\frac{\partial}{\partial\zeta} = \frac{\partial \phi}{\partial s} = -\frac{1}{\beta_0 c}\frac{\partial \phi}{\partial t} \label{derder}
\end{equation}


obtaining:
\begin{equation}
\textbf{F} 
=q \left( -\nabla \phi +\frac{\beta_0^2}{\gamma_0}\frac{\partial \phi}{\partial \zeta}\hat{\textbf{i}}_s
 + \beta  \beta_0\nabla \phi - \frac{\beta \beta_0}{\gamma_0} \frac{\partial \phi}{\partial \zeta} \hat{\textbf{i}}_s
  \right)
\end{equation}


Reorganizing:
\begin{equation}
\textbf{F} 
=  -q(1-\beta  \beta_0)\nabla \phi -\frac{\beta_0(\beta-\beta_0)}{\gamma_0}\frac{\partial \phi}{\partial \zeta}\hat{\textbf{i}}_s
\end{equation}

Explicit dependencies:
\begin{align}
F_x(x, y, \zeta(t)) &=  -q(1-\beta  \beta_0) \frac{\partial \phi}{\partial x}(x, y, \zeta(t))\\
F_y(x, y, \zeta(t)) &=  -q(1-\beta  \beta_0) \frac{\partial \phi}{\partial y}(x, y, \zeta(t))\\
F_z(x, y, \zeta(t)) &=  -q\left(1-\beta  \beta_0 -\frac{\beta_0(\beta-\beta_0)}{\gamma_0}\right) \frac{\partial \phi}{\partial \zeta}(x, y, \zeta(t))
\end{align}

Over the single interaction we neglect the particle slippage:
\begin{align}
&\beta = \beta_0\\
&\zeta(t) = \zeta
\end{align}

gives the following simplification:
\begin{align}
F_x(x, y, \zeta) &=  -q(1-\beta_0^2) \frac{\partial \phi}{\partial x}(x, y, \zeta)\\
F_y(x, y, \zeta) &=  -q(1-\beta_0^2) \frac{\partial \phi}{\partial y}(x, y, \zeta)\\
F_z(x, y, \zeta) &=  -q (1-\beta_0^2) \frac{\partial \phi}{\partial \zeta}(x, y, \zeta)
\end{align}

In this way the force over the single interaction becomes independent on time and therefore we can compute the kicks simply as:
\begin{equation}
\Delta \textbf{P} = \frac{L}{\beta_0 c}\textbf{F} 
\end{equation}

from which we can compute the kicks on the normalized momenta ($P_0=m_0\beta_0\gamma_0c$):

\begin{align}
\Delta p_x = \frac{\Delta P_x} {P}= -\frac{qL (1-\beta_0^2)}{ m_0\gamma_0\beta_0^2 c^2} \frac{\partial \phi}{\partial x}\left(x, y,\zeta\right)\label{dpx}\\
\Delta p_y = \frac{\Delta P_x} {P}= -\frac{qL (1-\beta_0^2)}{ m_0\gamma_0\beta_0^2 c^2} \frac{\partial \phi}{\partial y}\left(x, y,\zeta\right)\label{dpy}
\end{align}




\begin{thebibliography}{99}

\bibitem{benevento} G. Iadarola and G. Rumolo, ``Electron cloud effects`, proceedings of the ICFA Mini-Workshop on Impedances and Beam Instabilities in Particle Accelerators, Benevento, Italy, 2018.

\bibitem{rumolo_ruggiero} G. Rumolo, F. Ruggiero, and F. Zimmermann, ``Simulation of the electron-cloud build up and its consequences on heat load, beam stability, and diagnostics'', Phys. Rev. ST Accel. Beams 4, 012801, 2001.

\bibitem{jackson} J.D. Jackson, ``Classical electrodynamics'',  Wiley New York,  1999.

\bibitem{griffith} D. Griffith, ``Introduction to electrodynamics'', Prentice Hall, 1999.

\bibitem{zimmermann} F. Zimmermann, “A Simulation Study of Electron-Cloud Instability and Beam- Induced Multipacting in the LHC”, CERN LHC Project Report 95, SLAC-PUB- 7425 (1997).

\bibitem{headtail} G. Rumolo and F. Zimmermann, ``Practical user guide for HEADTAIL'', SL-Note-2002-036-AP.

\bibitem{gianni_thesis} G. Iadarola, ``Electron cloud studies for CERN particle accelerators and simulation code development'', CERN-THESIS-2014-047, 2014.

\bibitem{Iadarola:IPAC17-THPAB043}
	G.~Iadarola, E.~Belli, K.~S.~B.~Li, L.~Mether, A.~Romano, and G.~Rumolo,
``Evolution of Python Tools for the Simulation of Electron Cloud Effects'' proceedings of the 8th International Particle Accelerator Conference (IPAC'17), Copenhagen, Denmark, May 2017.

\bibitem{sixtrack} R. De Maria, A. Mereghetti, M. Fitterer, M. Fjellstrom, A. Patapenka, ``Sixtrack physics manual'', \url{http://sixtrack.web.cern.ch/SixTrack/docs/physics_manual.pdf}.

\bibitem{maxwellpinch} G.~Iadarola, ``Properties of the electromagnetic fields generated by a circular-symmetric e-cloud pinch in the ultra-relativistic limit'', CERN-ACC-NOTE-2019-0017.



%\bibitem{slides} G. Iadarola, R. De Maria, Y. Papaphilippou, ``Modelling and implementation of the ``6D'' beam-beam interaction'', CERN-ACC-SLIDES-2018-001, 2018 (version with animantions at \url{https://indico.cern.ch/event/684338/}).



\end{thebibliography}



\end{document}

