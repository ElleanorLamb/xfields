

\documentclass[a4paper,12pt]{report}

\usepackage{graphicx}
\usepackage{amssymb}
\usepackage{amsmath}
\usepackage{amsfonts}

\usepackage{placeins}


\usepackage[english]{babel}
\usepackage{booktabs}
\usepackage{stfloats}
\usepackage[T1]{fontenc}

\usepackage[titletoc,title]{appendix}

\usepackage{float}
\restylefloat{table}

\usepackage{longtable}

\usepackage{xcolor,colortbl}

\usepackage{mathpazo} % Palatino
\usepackage{avant}    % Avant Garde

\usepackage[margin=2.6cm, bottom=3.cm, top=3.cm, portrait]{geometry}\usepackage{caption}\usepackage[a4paper]{hyperref}\usepackage{adjustbox}\usepackage{enumitem}

 \usepackage[stable]{footmisc}

\renewcommand{\vec}[1]{\mathbf{#1}}

\setlength\parindent{0pt}

\newcommand{\HRule}[1]{\hfill \rule{0.2\linewidth}{#1}}

\makeatletter
\newcommand\primitiveinput[1]
{\@@input #1 }
\makeatother

\definecolor{Gray}{gray}{0.90}

\renewcommand\thesection{\arabic{section}}

%\renewcommand\appendixname{Appendix}
%\renewcommand\appendixpagename{Appendix}

\begin{document}


\thispagestyle{empty} % Remove page numbering on this page

%----------------------------------------------------------------------------------------
%	TITLE SECTION
%----------------------------------------------------------------------------------------

\colorbox{Gray}{
	\parbox[t]{1.0\linewidth}{
		\centering \fontsize{28pt}{40pt}\selectfont % The first argument for fontsize is the font size of the text and the second is the line spacing - you may need to play with these for your particular title
		\vspace*{0.7cm} % Space between the start of the title and the top of the grey box
		
		\hfill Xfields \\
		\hfill physics manual

%		\hfill Beam Screens\\
		
		\vspace*{0.7cm} % Space between the end of the title and the bottom of the grey box
	}
}

%----------------------------------------------------------------------------------------

%\vfill
%                \hfill {\huge Bunch length 1.0 ns}
                \vfill

%----------------------------------------------------------------------------------------
%	AUTHOR NAME AND INFORMATION SECTION
%----------------------------------------------------------------------------------------

{\centering \large 
\hfill Giovanni Iadarola \\
%\hfill \\
\hfill CERN - Geneva, Switzerland

\HRule{1pt}} % Horizontal line, thickness changed here

%----------------------------------------------------------------------------------------

\clearpage % Whitespace to the end of the page

\newpage

\renewcommand*{\arraystretch}{1.4}


\section{FFT method}

We illustrate the method in a single dimension then extend to multiple dimensions.



\section{Space charge}



We assume that the bunch travels rigidly along $s$ with velocity $\beta_0 c$:
\begin{align}
&\rho\left(x, y, s, t\right) = \rho_0\left(x, y, s - \beta_0 ct\right) \label{rhorho0}\\
&\textbf{J}\left(x, y, s, t\right) = \beta_0c\, \rho_0\left(x, y, s - \beta_0 ct\right)  \hat{\textbf{i}}_s \label{JJ0}
\end{align}

We define an auxiliary variable $\zeta$ as the position along the bunch:
\begin{equation}
\zeta = s -\beta_0 c t \, .\label{zetadef}
\end{equation}
We call $K$ the lab reference frame in which we have defined all equations above, and we introduce a boosted frame $K'$ moving rigidly with the reference particle.
The coordinates in the two systems are related by a Lorentz transformation~\cite{jackson}:
\begin{align}
ct' &= \gamma_0 \left(ct -\beta_0 s \right)\label{lorA}\\
x' &= x\label{lorX}\\
y' &= y\label{lorY}\\
s' &= \gamma_0 \left(s -\beta_0 ct \right) = \gamma_0 \zeta\label{lorB}
\end{align}
The  corresponding inverse transformation is:
\begin{align}
ct &= \gamma_0 \left(ct' +\beta_0 s' \right)\label{lorC}\\
x &= x'\label{lorXinv}\\
y &= y'\label{lorYinv}\\
s &= \gamma_0 \left(s' +\beta_0 ct' \right)\label{lorD}
\end{align}



The quantities $\left(c \rho, J_x, J_y, J_s\right)$ form a Lorentz 4-vector and therefore they are transformed between $K$ and $K'$ by relationships similar to the Eqs.~\ref{lorA}-\ref{lorY}~\cite{jackson}:
\begin{align}
c\rho' \left(\textbf{r'}, t'\right)\ &= \gamma_0 \left[c \rho  \left(\textbf{r}\left(\textbf{r'}, t'\right), t\left(\textbf{r'}, t'\right)\right) -\beta_0 J_s \left(\textbf{r}\left(\textbf{r'}, t'\right), t\left(\textbf{r'}, t'\right)\right) \right]\label{lorrho}\\
J_s' \left(\textbf{r'}, t'\right)\ &= \gamma_0 \left[J_s  \left(\textbf{r}\left(\textbf{r'}, t'\right), t\left(\textbf{r'}, t'\right)\right) -\beta_0 c \rho \left(\textbf{r}\left(\textbf{r'}, t'\right), t\left(\textbf{r'}, t'\right)\right) \right]\label{lorjs}
\end{align}
where the transformations $\textbf{r}\left(\textbf{r'}, t'\right)$ and $t\left(\textbf{r'}, t'\right)$ are defined by Eqs.~\ref{lorC} and~\ref{lorD} respectively. The transverse components $J_x$ and $J_y$ of the current vector are invariant for our transformation, and are anyhow zero in our case.

Using Eq.\,\ref{JJ0} these become:
\begin{align}
\rho' \left(\textbf{r'}, t'\right)\ &= \frac{1}{\gamma_0}\rho\left(\textbf{r}\left(\textbf{r'}, t'\right), t\left(\textbf{r'}, t'\right)\right)
\\
J_s' \left(\textbf{r'}, t'\right)\ & = 0
\end{align}

Using Eqs.~\ref{rhorho0} and~\ref{lorC}-\ref{lorYinv}, we obtain:
\begin{equation}
\rho  \left(x', y', s(s', t'), t(s', t')\right) = \rho_0  \left(x', y', s(s', t') - \beta_0 c\,t(s', t')\right)
\end{equation}

From Eq.~\ref{lorB} we get:
\begin{equation}
s(s', t')- \beta_0 c\,t(s', t') = \frac{s'}{\gamma_0} 
\end{equation}
where the coordinate $t' $ has disappeared.

We can therefore write:
\begin{equation}
\rho' \left(x', y', s', t'\right) =   \frac{1}{\gamma_0} \rho_0  \left(x', y',  \frac{s'}{\gamma_0}\right)\label{rhoprimerho0}
\end{equation}

The electric potential in the bunch frame is solution of Poisson's equation:

\begin{equation}
\frac{\partial^2 \phi'}{\partial x'^2} +  \frac{\partial^2 \phi'}{\partial y'^2}+  \frac{\partial^2 \phi'}{\partial s'^2}= -\frac{\rho' (x', y', s')}{\varepsilon_0}
\end{equation}

From Eq.~\ref{rhoprimerho0} we can write:
\begin{equation}
\frac{\partial^2 \phi'}{\partial x'^2} +  \frac{\partial^2 \phi'}{\partial y'^2}+  \frac{\partial^2 \phi'}{\partial s'^2}= -\frac{1}{\gamma_0\varepsilon_0}  \rho_0 \left(x', y', \frac{s'}{\gamma_0}\right)\label{poissrho0}
\end{equation}

We now make the substitution:
\begin{equation}
\zeta = \frac{s'}{\gamma_0} \label{subst}
\end{equation}
obtained from Eq.~\ref{lorB}, which allows to rewrite Eq.~\ref{poissrho0} as:
\begin{equation}
\frac{\partial^2 \phi'}{\partial x^2} +  \frac{\partial^2 \phi'}{\partial y^2}+  \frac{1}{\gamma_0^2}\frac{\partial^2 \phi'}{\partial \zeta^2}=  -\frac{1}{\gamma_0\varepsilon_0}{\rho}_0 \left(x, y,\zeta\right) \label{modifpoiss}
\end{equation}
Here we have dropped the ``$'$'' sign from $x$ and $y$ as these coordinates are unaffected by the Lorentz boost.




The quantities $\left( \frac{\phi}{c}, A_x, A_y, A_s\right)$ form a Lorentz 4-vector, we can show that the $s$ component of the vector potential in the lab frame vanishes:
\begin{align}
\phi &= \gamma_0 \left( {\phi'} +  \beta_0 c A'_s\right)\\
A_s &= A'_s +\beta_0 \frac{\phi'}{c}
\end{align}
In the bunch frame the charges are at rest therefore $A'_x=A'_y=A'_z=0$ therefore:
\begin{align}
\phi &= \gamma_0 \phi'\label{phiphip}\\
A_s &= \beta_0 \frac{\phi'}{c} =  \frac{\beta_0}{\gamma_0c}\phi
\end{align}

Combining Eq.\,\ref{phiphip} with Eq.\,\ref{modifpoiss} we obtain the equation in $\phi$:
\begin{equation}
\frac{\partial^2 \phi}{\partial x^2} +  \frac{\partial^2 \phi}{\partial y^2}+  \frac{1}{\gamma_0^2}\frac{\partial^2 \phi}{\partial \zeta^2}=  -\frac{1}{\varepsilon_0}{\rho}_0 \left(x, y,\zeta\right) \label{modifpoiss_zeta}
\end{equation}


%\section{Interaction time}
%In the lab frame the particle moves with speed $\beta$:
%\begin{equation}
%s(t) = \zeta_p +\beta c t
%\end{equation}
%
%In the frame $K'$, the kinematic equation of the particle can be obtained by replacing Eqs.~\ref{lorC} and~\ref{lorD} into Eq.~\ref{st_tau}:
%\begin{equation}
%\gamma_0 \left(s' +\beta_0 ct' \right) = \zeta_p +\beta \gamma_0 \left(ct' +\beta_0 s' \right)
%\end{equation}
%
%Solving for $s'$ we obtain:
%\begin{equation}
%s' = -\beta \gamma c \tau = \gamma \zeta\label{sprimezeta}
%\end{equation}
%Of course for the reference particle we obtain $s' = 0$.
%We observe that \textbf{beam particles are at rest in the reference frame $K'$ and that the distance between them is increased by a factor $\gamma$ with respect to the lab frame $K$}.

%\section{Transverse kick on the beam particle}
%
%We now evaluate the change on the transverse momentum for a beam particle defined in the lab frame by its transverse coordinates $x$ and $y$ and by its delay $\tau$ with respect to the reference particle (or equivalently by its $\zeta$ coordinate, defined by Eq.~\ref{zetadef}).
%
%We have seen that in the frame $K'$ the particle is at rest and has longitudinal coordinate $s' = \gamma \zeta$ (see Eq.~\ref{sprimezeta}). 
%The x' component of the electric field $\textbf{E}'$ acting on P is given by (see Eqs.~\ref{potential} and~\ref{phiphiprime}):
%\begin{equation}
%E'_x = -\frac{\partial \phi'}{\partial x} = -\frac{1}{\gamma_0}\frac{\partial \phi}{\partial x} \label{Exprime}
%\end{equation}
%Again, we have dropped the ``$'$'' sign from $x$ and $y$ as these coordinates are unaffected by the Lorentz boost.
%
%
%The change in the x component of the momentum, which is an invariant for our Lorentz transformation, is given by :
%\begin{equation}
%\Delta P_x = \Delta P'_x = qE'_x T'
%\end{equation}
%
%Using Eqs.~\ref{Exprime} and~\ref{Tprime} we can write:
%\begin{equation}
%\Delta P_x = -\frac{qL}{\beta c} \frac{\partial \phi}{\partial x}\left(x, y,\zeta\right)
%\end{equation}
%
%Normalizing to the momentum of the reference particle:
%
%\begin{equation}
%\Delta p_x = \frac{\Delta P_x} {P}= -\frac{qL}{ m\gamma\beta^2 c^2} \frac{\partial \phi}{\partial x}\left(x, y,\zeta\right)\label{dpx}
%\end{equation}
%
%Similarly, for the $y$-direction we can write: 
%\begin{equation}
%\Delta p_y = \frac{\Delta P_y} {P}= -\frac{qL}{ m\gamma\beta^2 c^2} \frac{\partial \phi}{\partial y}\left(x, y,\zeta\right)\label{dpy}
%\end{equation}


\section{Lorentz force}


We stay in the thin lens approximation so we approximate the velocity vector of the particle as:
\begin{equation}
\textbf{v} = \beta c\, \hat{\textbf{i}}_s
\end{equation}

We want to compute the Lorentz force acting on the particle:
\begin{equation}
\begin{split}
\textbf{F} &=q \left( -\nabla \phi -\frac{\partial \textbf{A}}{\partial t}
 + \beta c \ \hat{\textbf{i}}_s \times {\left(\nabla \times \textbf{A} \right)} \right)\\
 &=q \left( -\nabla \phi -\frac{\beta_0}{\gamma_0 c}\frac{\partial \phi}{\partial t}\hat{\textbf{i}}_s
 + \beta c \ \hat{\textbf{i}}_s \times {\left(\nabla \times \textbf{A} \right)} \right)
 \end{split}
\end{equation}

We compute the vector product:
\begin{align}
\begin{split}
\hat{\textbf{i}}_s \times \left(\nabla \times \textbf{A}\right) &= \left(\frac{\partial A_s}{\partial x} - \frac{\partial A_x}{\partial s} \right) \hat{\textbf{i}}_x + \left(\frac{\partial A_s}{\partial y} - \frac{\partial A_y}{\partial s} \right) \hat{\textbf{i}}_y\\
 &= \left(\frac{\partial A_s}{\partial x} - \frac{\partial A_x}{\partial s} \right) \hat{\textbf{i}}_x + \left(\frac{\partial A_s}{\partial y} - \frac{\partial A_y}{\partial s} \right) \hat{\textbf{i}}_y + \underbrace{\left(\frac{\partial A_s}{\partial s} - \frac{\partial A_s}{\partial s} \right)}_{=0} \hat{\textbf{i}}_s\\
 &= \nabla A_s - \frac{\partial \textbf{A}}{\partial s} 
\end{split} 
\end{align}

We replace:
\begin{equation}
\textbf{F} 
=q \left( -\nabla \phi -\frac{\beta_0}{\gamma_0 c}\frac{\partial \phi}{\partial t}\hat{\textbf{i}}_s
 + \beta  \beta_0\nabla \phi - \frac{\beta \beta_0}{\gamma_0} \frac{\partial \phi}{\partial s} \hat{\textbf{i}}_s
  \right)
\end{equation}

The potentials will have the same form as the sources (this can be shown explicitly using the Lorentz transformations):
\begin{equation}
\phi(x, y, s, t) = \phi\left(x, y, t - \frac{s}{\beta_0 c}\right)
\end{equation}
For a function in this form we can write:
\begin{equation}
 \frac{\partial \phi}{\partial s} = 
\frac{\partial}{\partial\zeta} 
 = -\frac{1}{\beta_0 c}\frac{\partial \phi}{\partial t} \label{derder}
\end{equation}


obtaining:
\begin{equation}
\textbf{F} 
=q \left( -\nabla \phi +\frac{\beta_0^2}{\gamma_0}\frac{\partial \phi}{\partial \zeta}\hat{\textbf{i}}_s
 + \beta  \beta_0\nabla \phi - \frac{\beta \beta_0}{\gamma_0} \frac{\partial \phi}{\partial \zeta} \hat{\textbf{i}}_s
  \right)
\end{equation}


Reorganizing:
\begin{equation}
\textbf{F} 
=  -q(1-\beta  \beta_0)\nabla \phi -\frac{\beta_0(\beta-\beta_0)}{\gamma_0}\frac{\partial \phi}{\partial \zeta}\hat{\textbf{i}}_s
\end{equation}

Explicit dependencies:
\begin{align}
F_x(x, y, \zeta(t)) &=  -q(1-\beta  \beta_0) \frac{\partial \phi}{\partial x}(x, y, \zeta(t))\\
F_y(x, y, \zeta(t)) &=  -q(1-\beta  \beta_0) \frac{\partial \phi}{\partial y}(x, y, \zeta(t))\\
F_z(x, y, \zeta(t)) &=  -q\left(1-\beta  \beta_0 -\frac{\beta_0(\beta-\beta_0)}{\gamma_0}\right) \frac{\partial \phi}{\partial \zeta}(x, y, \zeta(t))
\end{align}

Over the single interaction we neglect the particle slippage:
\begin{align}
&\beta = \beta_0\\
&\zeta(t) = \zeta
\end{align}
(in any case one would need to take into account also the dispersion in order to have the right slippage).

gives the following simplification:
\begin{align}
F_x(x, y, \zeta) &=  -q(1-\beta_0^2) \frac{\partial \phi}{\partial x}(x, y, \zeta)\\
F_y(x, y, \zeta) &=  -q(1-\beta_0^2) \frac{\partial \phi}{\partial y}(x, y, \zeta)\\
F_z(x, y, \zeta) &=  -q (1-\beta_0^2) \frac{\partial \phi}{\partial \zeta}(x, y, \zeta)
\end{align}

In this way the force over the single interaction becomes independent on time and therefore we can compute the kicks simply as:
\begin{equation}
\Delta \textbf{P} = \frac{L}{\beta_0 c}\textbf{F} 
\end{equation}

from which we can compute the kicks on the normalized momenta ($P_0=m_0\beta_0\gamma_0c$):

\begin{align}
\Delta p_x = \frac{m_0}{m}\frac{\Delta P_x} {P_0}= -\frac{qL (1-\beta_0^2)}{ m\gamma_0\beta_0^2 c^2} \frac{\partial \phi}{\partial x}\left(x, y,\zeta\right)\label{dpx}\\
\Delta p_y = \frac{m_0}{m}\frac{\Delta P_y} {P_0}= -\frac{qL (1-\beta_0^2)}{ m\gamma_0\beta_0^2 c^2} \frac{\partial \phi}{\partial y}\left(x, y,\zeta\right)\label{dpy}\\
\Delta \delta \simeq \Delta p_z = \frac{m_0}{m}\frac{\Delta P_z} {P_0}= -\frac{qL (1-\beta_0^2)}{ m\gamma_0\beta_0^2 c^2} \frac{\partial \phi}{\partial \zeta}\left(x, y,\zeta\right)\label{dpz}
\end{align}

Of your beam includes particles of different species (tracking of fragments), note that heree $q$ is the charge of the kicked particle while $m_0$ is the mass of the reference particle.


\subsection{2.5D approximation}
For large enough values of $\gamma_0$, Eq.~\ref{modifpoiss} can be approximated by:
\begin{equation}
\frac{\partial^2 \phi}{\partial x^2} +  \frac{\partial^2 \phi}{\partial y^2} = -\frac{1}{\varepsilon_0}{\rho}_0 \left(x, y,\zeta\right) \label{2dpoiss}
\end{equation}
which means that we can solve a simple 2D problem for each beam slice (identified by its $\zeta$).


\subsection{Modulated 2D}

Often the beam distribution can be factorized as:
\begin{equation}
\rho_0(x,y,\zeta) = Nq_0\lambda_0(\zeta)\rho_\perp(x,y) 
\end{equation}
where:
\begin{align}
&\int \lambda_0(z) \,dz = 1\\
&\int \rho_\perp(x,y) \,dx\,dy = 1
\end{align}
In this case the potential can be factorized as:
\begin{equation}
\phi(x,y,\zeta) = q_0\lambda_0(\zeta)\phi_\perp(x,y) 
\end{equation}

where $\phi_\perp(x,y)$ is the solution of the following 2D Poisson equation:
\begin{equation}
\frac{\partial^2 \phi_\perp}{\partial x^2} +  \frac{\partial^2 \phi_\perp}{\partial y^2} = -\frac{1}{\varepsilon_0}{\rho}_\perp \left(x, y\right) \label{2dpoisspeerp}
\end{equation}

The kick can be expressed as:
\begin{align}
\Delta p_x &= \frac{m_0}{m}\frac{\Delta P_x} {P_0}= -\frac{qq_0NL (1-\beta_0^2)}{ m\gamma_0\beta_0^2 c^2} \,\lambda_0(\zeta)\,\frac{\partial \phi}{\partial x}\left(x, y\right)\label{dpx_mod}\\
\Delta p_y &= \frac{m_0}{m}\frac{\Delta P_y} {P_0}= -\frac{qq_0NL (1-\beta_0^2)}{ m\gamma_0\beta_0^2 c^2} \,\lambda_0(\zeta)\, \frac{\partial \phi}{\partial y}\left(x, y\right)\label{dpy_mod}\\
\Delta \delta \simeq \Delta p_z &= \frac{ m_0} {m}\frac{\Delta P_z} {P_0}= -\frac{qq_0NL (1-\beta_0^2)}{ m\gamma_0\beta_0^2 c^2} \,\frac{d\lambda_0}{d\zeta}(\zeta)\,\phi\left(x, y\right)\label{dpz_mod}
\end{align}

\section{FFT solver}

We start from a 1D case for illustration and then we generalize.

We assume free space. The potential can be written as the convolution of a Green function with the charge distribution:
\begin{equation}
\phi(x) = \int_{-\infty}^{+\infty} \rho(x')\,G(x-x') dx'
\label{eq:conv}
\end{equation}

We assume that the source is limited to a region of measure $L$:
\begin{equation}
\rho(x) = \rho(x)\,\Pi\left(\frac{x}{L}\right)
\label{eq:rholim}
\end{equation}
where $\Pi(x)$ is a rectangular window defined as:
\begin{equation}
\Pi(x) = 
\begin{cases}
1\quad\text{for } -\frac{1}{2}<x<\frac{1}{2}\\
0\quad\text{elsewhere}
\end{cases}
\end{equation}

We are interested in the electric potential only the region occupied by the sources, so we can compute:
\begin{equation}
\phi_L(x) = \phi(x) \Pi\left(\frac{x}{L}\right)
\label{eq:philim}
\end{equation}

We replace Eq.\,\eqref{eq:rholim} and Eq.\,\eqref{eq:philim} into Eq.\eqref{eq:conv}, obtaining:
\begin{equation}
\phi_L(x) = \Pi\left(\frac{x}{L}\right)
\int_{-\infty}^{+\infty} 
\Pi\left(\frac{x'}{L}\right)
\rho(x')\,G(x-x') dx'
\end{equation}
We change variable into $x'' = x - x'$:
\begin{equation}
\phi_L(x) = 
\int_{-\infty}^{+\infty} 
\Pi\left(\frac{x}{L}\right)
\Pi\left(\frac{x-x''}{L}\right)
\rho(x-x'')\,G(x'') dx''
\label{eq:conv1}
\end{equation}
The integrand vanishes outside the set of the $(x, x'')$ defined by:
\begin{equation}
\begin{cases}
-\frac{L}{2} < x <\frac{L}{2}\\
-\frac{L}{2} < (x"-x) <\frac{L}{2}
\end{cases}
\end{equation}

Combining the two equations we obtain:
\begin{equation}
-L<-\frac{L}{2}+x < x'' <\frac{L}{2}+x<L
\end{equation}
i.e. the integrand is zero for $-L<x''<L$.
Therefore in equation \eqref{eq:conv1} we can replace the $G(x'')$ with its truncated version:
\begin{equation}
G_{2L}(x'') = G(x'')\,\Pi
\left(
\frac{x''}{2L}
\right)
\end{equation}
obtaining:
\begin{equation}
\phi_L(x) = 
\int_{-\infty}^{+\infty} 
\Pi\left(\frac{x}{L}\right)
\Pi\left(\frac{x-x''}{L}\right)
\rho(x-x'')\,G_{2L}(x'') dx''
\label{eq:conv2}
\end{equation}

Since the two window function force the integrand to zero outside the region $|x''|<L$ we can replace $G_{2L}(x'')$ with its replicated version:
\begin{equation}
G_{2LR}(x'') = \sum_{n=-\infty}^{+\infty}G_{2L}(x'-2nL) = \sum_{n=-\infty}^{+\infty}G(x''-2nL)\,\Pi
\left(
\frac{x''-2nL}{2L}
\right)
\label{eq:GRL}
\end{equation}
obtaining:
\begin{equation}
\phi_L(x) = 
\int_{-\infty}^{+\infty} 
\Pi\left(\frac{x}{L}\right)
\Pi\left(\frac{x-x''}{L}\right)
\rho(x-x'')\,G_{2LR}(x'') dx''
\end{equation}

We can go back to the initial coordinate by substituting $x'' = x-x'$:
\begin{equation}
\phi_L(x) = 
\Pi\left(\frac{x}{L}\right)
\int_{-\infty}^{+\infty} 
\rho(x')\,G_{2LR}(x-x') dx'
\end{equation}

This is a cyclic convolution, so we can proceed as followd. We split the integral:
\begin{equation}
\phi_L(x) = 
\Pi\left(\frac{x}{L}\right)
\sum_{n=-\infty}^{+\infty}
\int_{-\frac{L}{2} +2nL }^{+\frac{L}{2} +2nL } 
\rho(x')\,G_{2LR}(x-x') dx'
\label{eq:conv3}
\end{equation}
In each term I replace $x''' = x'+2nL$:
\begin{equation}
\phi_L(x) = 
\Pi\left(\frac{x}{L}\right)
\sum_{n=-\infty}^{+\infty}
\int_{-\frac{L}{2}  }^{+\frac{L}{2}  } 
\rho(x'''-2nL)\,G_{2LR}(x-x'''-2nL) dx'
\label{eq:conv4}
\end{equation}
But $G_{2LR}$ is periodic:
\begin{equation}
\begin{split}
\phi_L(x) &= 
\Pi\left(\frac{x}{L}\right)
\sum_{n=-\infty}^{+\infty}
\int_{-\frac{L}{2}  }^{+\frac{L}{2}  }
\rho(x'''-2nL)\,G_{2LR}(x-x''') dx'''\\
\\&=
\Pi\left(\frac{x}{L}\right)
\int_{-\frac{L}{2}  }^{+\frac{L}{2}  }
\sum_{n=-\infty}^{+\infty}
\rho(x'''-2nL)\,G_{2LR}(x-x''') dx'''
\end{split}
\label{eq:conv5}
\end{equation}

I can define a replicated version of $\rho(x)$:
\begin{equation}
\rho_{2LR}(x)= \sum_{n=-\infty}^{+\infty}
\rho(x-2nL)
\end{equation}

obtaining:
\begin{equation}
\phi_L(x) = 
\Pi\left(\frac{x}{L}\right)
\int_{0 }^{2L} 
\rho_{2LR}(x')\,G_{2LR}(x-x') dx'
\label{eq:conv6}
\end{equation}

The function:

\begin{equation}
\phi_{2LR}(x) = 
\int_{-\frac{L}{2}  }^{+\frac{L}{2}  }
\rho_{2LR}(x')\,G_{2LR}(x-x') dx'
\label{eq:confin}
\end{equation}
is periodic of period $2L$. From it the potential of interest can be simply calculated by selecting the  half period $[-L/2, L/2]$:
\begin{equation}
\phi_L(x) = 
\Pi\left(\frac{x}{L}\right)
\phi_{2LR}(x)
\label{eq:sel}
\end{equation}

We expand $\phi_{2LR}(x)$ in Fourier series:
\begin{equation}
\phi_{2LR}(x) = \sum_{k=-\infty}^{+\infty} \tilde{\phi}_k\, e^{j2\pi k \frac{x}{2L}}
\label{eq:phifour}
\end{equation}
where the Fourier coefficient are given by:

\begin{equation}
\tilde{\phi}_k = \frac{1}{2L}\int_0^{2L} \phi_{2LR}(x)\, e^{-j2\pi k \frac{x}{2L}} \, dx
\label{eq:phik}
\end{equation}

We replace Eq.\,\eqref{eq:confin} into Eq.\,\eqref{eq:phik} obtaining:
\begin{equation}
\hat{\phi}_k = \frac{1}{2L}\int_0^{2L} \int_{0 }^{2L} 
\rho_{2LR}(x')\,G_{2LR}(x-x') \, e^{-j2\pi k \frac{x}{2L}} \,  dx'\, dx
\end{equation}

With the change of variable $x'' = x-x'$ we obtain:
\begin{equation}
\tilde{\phi}_k = 
\frac{1}{2L}
\int_{-\frac{L}{2}  }^{+\frac{L}{2}  } 
\rho_{2LR}(x') e^{-j2\pi k \frac{x'}{2L}}dx'\,
\int_{-\frac{L}{2}  }^{+\frac{L}{2}  }
\,G_{2LR}(x'') e^{-j2\pi k \frac{x''}{2L}}\,  \,  dx''
\end{equation}

where we recognize the Fourier coefficients of $\rho_{2LR}(x)$ and $\,G_{2LR}(x)$:
\begin{align}
\tilde{\rho}_k = \frac{1}{2L}
\int_{-\frac{L}{2}  }^{+\frac{L}{2}  }
 \rho_{2LR}(x)\, e^{-j2\pi k \frac{x}{2L}} \, dx \label{eq:rhok}\\
\tilde{G}_k = \frac{1}{2L}
\int_{-\frac{L}{2}  }^{+\frac{L}{2}  }
 G_{2LR}(x)\, e^{-j2\pi k \frac{x}{2L}} \, dx \label{eq:Gk}
\end{align}
obtaining simply:
\begin{equation}
\hat{\phi}_k = 2L \, \hat{G}_k \, \hat{\rho}_k
\label{eq:freqconv}
\end{equation}

I assume to have the functions $\rho_{2LR}(x)$ and  $G_{2LR}(x)$ sampled (or averaged) with step:
\begin{equation}
h_x = \frac{2L}{M} = \frac{L}{N}
\end{equation}

I can approximate the integrals in Eqs.\,\eqref{eq:rhok} and\,\eqref{eq:Gk} as:
\begin{align}
\tilde{\rho}_k = \frac{1}{M}\sum_{n=0}^{M-1} \rho_{2LR}(x_n)\, e^{-j2\pi  \frac{kn}{M}}  
= \frac{1}{M} \hat{\rho}_k
\label{eq:rhokfft}\\
\tilde{G}_k = \frac{1}{M}\sum_{n=0}^{M-1} G_{2LR}(x_n)\, e^{-j2\pi  \frac{kn}{M}} 
= \frac{1}{M} \hat{G}_k\label{eq:Gkfft}
\end{align}

where we recognize the Discrete Fourier Transforms:
\begin{align}
\hat{\rho}_k = \text{DFT}\left\{ \rho_{2LR}(x_n)\right\}\\
\hat{G}_k = \text{DFT}\left\{ G_{2LR}(x_n)\right\}
\end{align}



Using Eq.\,\eqref{eq:phifour} we can obtained a sampled version of $\phi(x)$:
\begin{equation}
\phi_{2LR}(x_n) = 
\sum_{n=0}^{M-1}  
\tilde{\phi}_k\, e^{j2\pi \frac{kn}{M}}
\label{eq:phifft}
\end{equation}
where we have assumed that $\phi(x)$ is sufficiently smooth to allow truncating the sum.


Using Eqs.\,\eqref{eq:rhokfft} and\,\eqref{eq:Gkfft} 
\begin{equation}
\phi_{2LR}(x_n) = 
2L \sum_{n=0}^{M-1}  
\tilde{G}_k \, \tilde{\rho}_k\, e^{j2\pi \frac{kn}{M}} 
= 
\frac{2L}{M^2}
\sum_{n=0}^{M-1}  
\hat{G}_k \, \hat{\rho}_k\, e^{j2\pi \frac{kn}{M}} 
\label{eq:phifftsimpl}
\end{equation}

This can be rewritten as:
\begin{equation}
\phi_{2LR}(x_n) = 
\frac{1}{M}
\sum_{n=0}^{M-1}  
(h_x\hat{G}_k) \, \hat{\rho}_k\, e^{j2\pi \frac{kn}{M}} 
=\text{DFT}^{-1}\left\{
(h_x\hat{G}_k) \, \hat{\rho}_k)
\right\}
\label{eq:invfft}
\end{equation}

In summary the potential at the grid nodes can be computed as follows:
\begin{enumerate}
\item We sample or average the Green function with step $h_x$ over the interval $[0, L]$, obtaining the sequence $G(x_n)$ for $n=0, ..., N-1$
\item We build its replicate version in the interval $[L, 2L]$ using the fact that: 
\begin{equation}
G_{2LR}(2L-x) = G(-x) = G(x)
\end{equation}
\end{enumerate}

\subsection{Extension to 3D}

In summary in 1D what one needs to do is:

Transform the sources and the Green function:
\begin{align}
\hat{\rho}_k = \frac{1}{M}\sum_{n=0}^{M-1} \rho_{2LR}(x_n)\, e^{j2\pi  \frac{kn}{M}} \\
\hat{G}_k = \frac{1}{M}\sum_{n=0}^{M-1} G_{2LR}(x_n)\, e^{j2\pi  \frac{kn}{M}}  
\end{align}

Do the product and antitransform:
\begin{equation}
\phi_{2LR}(x_n) = 
2L \sum_{n=0}^{M-1}  
\hat{G}_k \, \hat{\rho}_k\, e^{-j2\pi \frac{kn}{M}}
\end{equation}

Repeating the same reasoning along all directions one obtains:
\begin{align}
&\hat{\rho}_{k_x k_y k_z} = 
\frac{1}{M_x M_y M_z}
\sum_{n_x=0}^{M_x-1} e^{j 2\pi  \frac{k_x n_x}{M_x}} 
\sum_{n_y=0}^{M_y-1} e^{j 2\pi  \frac{k_y n_y}{M_y}}
\sum_{n_z=0}^{M_z-1} e^{j 2\pi  \frac{k_z n_z}{M_z}}\, 
\rho_{2LR}(x_{n_x}, y_{n_y}, z_{n_z})\\
&\hat{G}_{k_x k_y k_z} = 
\frac{1}{M_x M_y M_z}
\sum_{n_x=0}^{M_x-1} e^{j 2\pi  \frac{k_x n_x}{M_x}} 
\sum_{n_y=0}^{M_y-1} e^{j 2\pi  \frac{k_y n_y}{M_y}}
\sum_{n_z=0}^{M_z-1} e^{j 2\pi  \frac{k_z n_z}{M_z}}\, 
G_{2LR}(x_{n_x}, y_{n_y}, z_{n_z})
\end{align}

\begin{multline}
\phi_{2LR}(x_{n_x}, y_{n_y}, z_{n_z}) = \\
(2L_x)(2L_y)(2L_z)
\sum_{k_x=0}^{M_x-1} e^{-j 2\pi  \frac{k_x n_x}{M_x}} 
\sum_{k_y=0}^{M_y-1} e^{-j 2\pi  \frac{k_y n_y}{M_y}}
\sum_{k_z=0}^{M_z-1} e^{-j 2\pi  \frac{k_z n_z}{M_z}}\, 
\hat{G}_{k_x k_y k_z} \hat{\rho}_{k_x k_y k_z}
\end{multline}

The continuous convolution in 3D is written as 

\begin{equation}
\phi(x, y, z) = \iiint_{-\infty}^{+\infty} \rho(x', y', z')
   \,G(x-x', y-y', z-z')\,dx'\,dy'\,dz'
\end{equation}

where:
\begin{equation}
G(x, y, z) = \frac{1}{4\pi\varepsilon_0}\frac{1}{
\sqrt{x^2 +y^2 +z^2}
}
\end{equation}

\begin{thebibliography}{99}

\bibitem{benevento} G. Iadarola and G. Rumolo, ``Electron cloud effects`, proceedings of the ICFA Mini-Workshop on Impedances and Beam Instabilities in Particle Accelerators, Benevento, Italy, 2018.

\bibitem{rumolo_ruggiero} G. Rumolo, F. Ruggiero, and F. Zimmermann, ``Simulation of the electron-cloud build up and its consequences on heat load, beam stability, and diagnostics'', Phys. Rev. ST Accel. Beams 4, 012801, 2001.

\bibitem{jackson} J.D. Jackson, ``Classical electrodynamics'',  Wiley New York,  1999.

\bibitem{griffith} D. Griffith, ``Introduction to electrodynamics'', Prentice Hall, 1999.

\bibitem{zimmermann} F. Zimmermann, “A Simulation Study of Electron-Cloud Instability and Beam- Induced Multipacting in the LHC”, CERN LHC Project Report 95, SLAC-PUB- 7425 (1997).

\bibitem{headtail} G. Rumolo and F. Zimmermann, ``Practical user guide for HEADTAIL'', SL-Note-2002-036-AP.

\bibitem{gianni_thesis} G. Iadarola, ``Electron cloud studies for CERN particle accelerators and simulation code development'', CERN-THESIS-2014-047, 2014.

\bibitem{Iadarola:IPAC17-THPAB043}
	G.~Iadarola, E.~Belli, K.~S.~B.~Li, L.~Mether, A.~Romano, and G.~Rumolo,
``Evolution of Python Tools for the Simulation of Electron Cloud Effects'' proceedings of the 8th International Particle Accelerator Conference (IPAC'17), Copenhagen, Denmark, May 2017.

\bibitem{sixtrack} R. De Maria, A. Mereghetti, M. Fitterer, M. Fjellstrom, A. Patapenka, ``Sixtrack physics manual'', \url{http://sixtrack.web.cern.ch/SixTrack/docs/physics_manual.pdf}.

\bibitem{maxwellpinch} G.~Iadarola, ``Properties of the electromagnetic fields generated by a circular-symmetric e-cloud pinch in the ultra-relativistic limit'', CERN-ACC-NOTE-2019-0017.



%\bibitem{slides} G. Iadarola, R. De Maria, Y. Papaphilippou, ``Modelling and implementation of the ``6D'' beam-beam interaction'', CERN-ACC-SLIDES-2018-001, 2018 (version with animantions at \url{https://indico.cern.ch/event/684338/}).



\end{thebibliography}



\end{document}

